\documentclass{crudecv/crudecv}
\usepackage[utf8]{inputenc}
\usepackage{lipsum}
\usepackage{blindtext}
\usepackage{multicol}
\usepackage{hyperref}

\author{Eldar Hauge Tork\titleformat{\section}
{\normalfont\Large\bfseries}{\thesection}{1em}{}[{\titlerule[1pt]}]elsen}
\applicant{18/04/1997}{Ellsworthsgate 56 4019 Stavanger, Norway}{eldar.h.t@hotmail.com}{41429930}{https://github.com/eldarht}

\begin{document}
\maketitle

\section*{Interest areas}
Within the field of computer science, I would like to write computer programs with a focus on problem solving. Preferably on code that provide functionality with a well defined and unambiguous goal, where the input and output is well defined and assumptions about the end-users behaviour does not change the implementation. This could be back-end development, library functions, data-interpretation, sensor-functionality, complex algorithms or similar. I have tried several languages and could work with any of them, but i prefer compiled languages with type safety and a strict syntax.

\section*{Education}
\begin{experiences}
	\experience{Master of Science in Informatics}{
		specialization: Software
		Norges teknisk-naturvitenskapelige universitet (NTNU) 
	}{
		2019 autumn
	}{
		2021 spring
	}
	\experience{Bachelor in Programming  [Games \rule[-0.3ex]{0.2ex}{0.8em} App]}{
		specialization: Games 
		Norges teknisk-naturvitenskapelige universitet (NTNU)
		\textbf{With extra course:} IMT2291 Web Technology
	}{
		2016 autumn
	}{
		2019 spring
	}
\end{experiences}

\section*{Work experience}

\begin{experiences}
	\experience{Teaching assistant}{
		Giving feedback on student projects for IMT2243 - Software Engineering.
	}{
		2018 spring
	}{
		2018 autumn
	}
	\experience{Part-time at a postal service}{
		Sorting and checking postage on packages/letters.
	}{
		2013.09.01
	}{
		2018.02.28
	}
\end{experiences}


\section*{Extracurricular activity and volunteer work}
\begin{experiences}
	\experience{Student association board member}{
		At Login - student association for IT students at NTNU campus Gj\o vik. (Equivalent to Online in Trondheim)
	}{
		2016 autumn
	}{
		2019 spring
	}
	\experience{Course reference groups}{
		Connection point between course coordinator, students and NTNU quality assurance. I was part of the reference groups in 9 different courses: IMT1031, REA1101, IMT1082, IMT2243, REA1121, IMT2571, IMT2531, IMT3103 and TDT4165.
	}{
		2016 autumn
	}{
		2020 spring
	}
\end{experiences}


\section*{Training courses}
\begin{experiences}
	\experience{
		LAOS – Learning assistant training
	}{
		Learning to give better feedback and guidance.
	}{
		2018 spring
	}{ 
		2018 automn
	}
\end{experiences}

\section*{Languages}
\begin{itemize}
    \item Norwegian: Primary language
    \item English: Secondary language, fluent
\end{itemize}


\section*{Programming languages}
\begin{skills}
	\skill{C/C++/C\#}{4}
	\skill{Java}{4}
	\skill{Golang}{3}
	\skill{\LaTeX}{3}
	\skill{PHP}{2}
	\skill{Phyton}{2}
	\skill{JavaScript}{2}
	\skill{oz}{2}
	\skill{lisp}{1}LaTeX
	\skill{prolog}{1}
	\skill{Bash}{1}
	\skill{PowerShell}{1}
\end{skills}
\section*{Platform familiarity}
I have been using Linux as my main platform for more than five years and prefer it for work related activities. I have user-experience with macOS, Windows, Android and an older version of Amiga. I have written programs targeting Windows, Linux, Android and web.

\section*{Projects}
\textbf{CodebaseVisualizer3D:} \hfill \href{https://github.com/zohaib194/CodebaseVisualizer3D}{https://github.com/zohaib194/CodebaseVisualizer3D} \\
Bachelor-project with java/C++ parsing, Go api server and client withe Three.js/WebGL rendering.
\end{document}