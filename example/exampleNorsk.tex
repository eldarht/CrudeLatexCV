\documentclass[norsk]{../crudecv/crudecv}
\usepackage[utf8]{inputenc}
\usepackage{lipsum}
\usepackage{blindtext}
\usepackage{multicol}
\usepackage{hyperref}
\usepackage{xfrac}
\usepackage{glossaries}

\author{Eldar Hauge Torkelsen}
\applicant{18/04/1997}{Dalen 5 4340 Brye, Norge}{eldar.h.t@hotmail.com}{41429930}{https://github.com/eldarht}

\newacronym{ntnu}{NTNU}{Norges teknisk-naturvitenskapelige universitet}
\begin{document}
\maketitle

\section*{Profil}
Innebygget programvare utvikler med utdannelse innen programmering og
informatikk fra \gls*{ntnu}. Har interesse for å programmere godt definerte
grensesnitt, utfordrende algoritmer og støttende arkitekturer. Erfaren med
flere plattformer og programmeringsspråk, med en spesiell fasinasjon for
kompilerte programmeringsspråk med strikt syntaks.

\section*{Arbeidserfaring}
\begin{experiences}
  \experience{Utviklingsingeniør i Datarespons R\&D}{
    Jobber som konsulent innenfor innebygger programvare utvikling. Har i hovedsak
    jobbet på et prosjekt hos Zaptec. Orinalt omhandlet prosjektet integrering
    av Quectel-BG95, men ble utvidet til å involvere arbeid med SecureBoot og
    kryptering med esp-idf, OCPP, enhetsprovisjonering og mer.
  }{
    2021 høst
  }{
    nåværende
  }
  \experience{Lærings assistent}{
    Gi tilbakemeldinger på studenters prosjekt for IMT2243 -  Systemutvikling.
  }{
    2018 høst
  }{
    2018 vår
  }
  \experience{Deltidsjobb at på post terminal}{
    Sortering og sjekking av porto på pakker/brev.
  }{
    2013 høst
  }{
    2018 vår
  }
\end{experiences}

\section*{Utdanning}
\begin{experiences}
  \experience{Master i informatikk (\sfrac{112.5}{120} studiepoeng)}{
    spesialisering: Programvaresystemer - Fokuserer på interdisiplinære studier med IT\\
    Studieplass: \gls*{ntnu}
  }{
    2019 høst
  }{
    2021 vår
  }
  \experience{Bachelor i Programmering [Spill \rule[-0.3ex]{0.2ex}{0.8em}
    Applikasjoner] (\sfrac{190}{180} studiepoeng)}{
    spesialisering: Spill - Fokus på programmering innenfor komplekse og voksende applikasjoner.\\
    Studieplass: \gls*{ntnu} \\
    \textbf{Med ekstra fag:} IMT2291 WWW-teknologi
  }{
    2016 høst
  }{
    2019 vår
  }
  \experience{Hetland Videregående skole}{
    Valgfag: matematikk R1, og R2, informasjonsteknologi, kjemi, fysikk, spansk.
  }{
    2013 høst
  }{
    2016 vår
  }
\end{experiences}

\section*{Frivillig arbeid og tillitsverv}
\begin{experiences}
  \experience{Styremedlem i linjeforening}{
    I Login - Linjeforeningen for IT studenter ved \gls*{ntnu} campus Gj\o vik. (tilsvarende Online i Trondheim)
  }{
    2016 vår
  }{
    2019 høst
  }
  \experience{Referanse gruppe}{
    Tilknytningsledd mellom fagansvarlig, studenter og \gls*{ntnu} kvalitetssikring. Tok del i referansegruppene i 9 forskjellige fag: IMT1031, REA1101, IMT1082, IMT2243, REA1121, IMT2571, IMT2531, IMT3103 and TDT4165.
  }{
    2016 vår
  }{
    2020 høst
  }
\end{experiences}

\section*{Kurs og sertifikater}
\begin{experiences}
  \experience{LAOS – Læringsassistent kurs}{
    Lærte å gi bedre tilbakemeldinger og veiledning.
  }{
    2018 vår
  }{
    2018 høst
  }
\end{experiences}

\section*{Språk}
\begin{itemize}
\item Norsk: Hovedmål
\item English: flytende
\end{itemize}


\section*{Programmeringsspråk}
\begin{skills}
  \skill{C/C++/C\#}{5}
  \skill{Java}{5}
  \skill{Golang}{4}
  \skill{\LaTeX}{4}
  \skill{PHP}{3}
  \skill{Phyton}{3}
  \skill{JavaScript}{3}
  \skill{oz}{3}
  \skill{Bash}{3}
  \skill{lisp}{2}
  \skill{PowerShell}{2}
  \skill{prolog}{1}
\end{skills}


\section*{Plattform kunnskaper}
Har brukt Linux som min hovedplattform i mer enn 7 år og foretrekker det for
arbeidsrelaterte aktiviteter. Har brukererfaring med macOS, Windows, Android
og en eldre versjon av Amiga. Har skrevet programmer rettet mot Windows,
Linux, Yocto, Zephyr, esp-idf Android og web.

\section*{Prosjekt}
\textbf{CodebaseVisualizer3D:} \hfill \href{https://github.com/zohaib194/CodebaseVisualizer3D}{https://github.com/zohaib194/CodebaseVisualizer3D} \\
Bachelor-prosjekt med java/C++ parsing, Go API server og klient med Three.js/WebGL rendering.

\end{document}
